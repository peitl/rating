\documentclass[preprint,12pt]{article}
\usepackage[T1]{fontenc}
\usepackage{graphicx}
\usepackage{amsmath}
\usepackage{verbatim}

\newcommand{\ch}{\v{c}}
\newcommand{\tj}{\'c}
\newcommand{\ndi}{\chi_m^e}
\newcommand{\sd}{\chi_\Sigma^e}
\newcommand{\E}{\mathcal{E}}

\DeclareMathOperator{\MSE}{MSE}
\DeclareMathOperator{\MAE}{MAE}
\DeclareMathOperator{\avg}{avg}

\begin{document}

\begin{titlepage}
\title{
	{Statistical comparison of the proposed and the current rating system}
	}
\author{Bojan Vu\ch kovi\tj}
\end{titlepage}
\maketitle

\section{Tables}

We have tested behaviour of the proposed rating system
on the available $95$ solving tournaments
held from October 2015 till July 2018.
Starting rating list is the official October 2015 rating list taken from the WFCC site.
Statistics obtain for the current rating system are given in Table~\ref{old_system}.
Value given in columns $\avg R_{k}$ designates the average rating of the $k$
best rated active solvers from specified lists.

\begin{table}[h]
\begin{center}
\begin{tabular}{|l|c|c|c|c|}
\hline
Rating list & $\avg (R_{20})$ & $\avg (R_{50})$ & $\avg (R_{100})$ & $\avg (R_{200})$\\
\hline
October $2015$ & $2623.60$ & $2520.54$ & $2423.57$ & $2297.23$\\
\hline
January $2016$ & $2622.23$ & $2523.21$ & $2426.81$ & $2300.27$\\
\hline
July $2016$ & $2616.35$ & $2522.00$ & $2427.26$ & $2306.63$\\
\hline
January $2017$ & $2611.25$ & $2520.37$ & $2428.45$ & $2309.01$\\
\hline
July $2017$ & $2606.80$ & $2518.06$ & $2426.89$ & $2309.09$\\
\hline
January $2018$ & $2607.81$ & $2519.26$ & $2427.56$ & $2309.41$\\
\hline
July $2018$ & $2596.24$ & $2507.52$ & $2421.83$ & $2310.80$\\
\hline
\end{tabular}
\caption{Behaviour of the current rating system}
\label{old_system}
\end{center}
\end{table}

Let $c_1$ be the corrective coefficient for solvers with rating below $2400$,
and $c_2$ be the corrective coefficient for solvers with rating below $2200$.
We have tested two variations for corrective coefficients $c_i$.
The first is $c_1=2$ and $c_2=3$ (Table~\ref{new_system_1}),
and the second is $c_1=2$ and $c_2=4$ (Table~\ref{new_system_2}).
\begin{table}[h]
\begin{center}
\begin{tabular}{|l|c|c|c|c|}
\hline
Rating list & $\avg (R_{20})$ & $\avg (R_{50})$ & $\avg (R_{100})$ & $\avg (R_{200})$\\
\hline
October $2015$ & $2623.60$ & $2520.54$ & $2423.57$ & $2297.23$\\
\hline
January $2016$ & $2620.33$ & $2522.31$ & $2424.30$ & $2300.99$\\
\hline
July $2016$ & $2617.65$ & $2526.44$ & $2433.43$ & $2313.07$\\
\hline
January $2017$ & $2609.95$ & $2524.51$ & $2439.99$ & $2319.90$\\
\hline
July $2017$ & $2604.32$ & $2520.90$ & $2436.51$ & $2320.13$\\
\hline
January $2018$ & $2602.18$ & $2518.11$ & $2436.64$ & $2322.74$\\
\hline
July $2018$ & $2595.75$ & $2511.49$ & $2434.26$ & $2327.32$\\
\hline
\end{tabular}
\caption{Behaviour of the proposed rating system with $c_1=2$ and $c_2=3$}
\label{new_system_1}
\end{center}
\end{table}

\begin{table}[h]
\begin{center}
\begin{tabular}{|l|c|c|c|c|}
\hline
Rating list & $\avg (R_{20})$ & $\avg (R_{50})$ & $\avg (R_{100})$ & $\avg (R_{200})$\\
\hline
October $2015$ & $2623.60$ & $2520.54$ & $2423.57$ & $2297.23$\\
\hline
January $2016$ & $2620.33$ & $2522.31$ & $2424.83$ & $2302.32$\\
\hline
July $2016$ & $2618.87$ & $2531.25$ & $2438.63$ & $2318.71$\\
\hline
January $2017$ & $2613.14$ & $2530.14$ & $2444.97$ & $2326.50$\\
\hline
July $2017$ & $2609.01$ & $2526.82$ & $2445.58$ & $2330.00$\\
\hline
January $2018$ & $2608.68$ & $2526.20$ & $2449.29$ & $2334.99$\\
\hline
July $2018$ & $2603.68$ & $2521.33$ & $2448.21$ & $2341.99$\\
\hline
\end{tabular}
\caption{Behaviour of the proposed rating system with $c_1=2$ and $c_2=4$}
\label{new_system_2}
\end{center}
\end{table}

Though there is still a certain rating deflation for top $20$ and top $50$
solvers shown in Table~\ref{new_system_1}, this might be a better solution
than the one shown in Table~\ref{new_system_2}.
There is an inflation of average ratings of top $100$ and solvers $200$,
which, in the second case, might become a problem after a couple of years.
After certain time, it is expected that rating deflation of top $20$
solvers completely stops as ratings of solvers with lower rating increase.
Thus, $c_1=2$ and $c_2=3$ might be a better choice from these two options.

\end{document}